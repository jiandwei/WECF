% ==================== 基础包 ====================
% \usepackage[utf8]{inputenc}
% \usepackage[T1]{fontenc}
% \usepackage[english]{babel}
\usepackage{amsmath,amsfonts,amssymb,amsthm}
\usepackage{mathtools}
\usepackage[colorinlistoftodos]{todonotes}
\usepackage{xeCJK}

% ==================== 页面设置 ====================
\usepackage[margin=2cm]{geometry}
\usepackage{setspace}
\onehalfspacing

% ==================== 字体设置(Windows) ====================
% 英文字体
\setmainfont{Times New Roman}
% 中文字体(推荐支持粗体的思源宋体)
\setCJKmainfont{Source Han Serif SC}
\setCJKmonofont{Source Han Mono SC}

% ==================== 图表支持 ====================
\usepackage{graphicx}
\usepackage{float}
\usepackage{caption}
\usepackage{subcaption}
\usepackage{booktabs}
\usepackage{multirow}
\usepackage{array}

% ==================== 颜色和链接 ====================
\usepackage{xcolor}
\usepackage[colorlinks=true,
    linkcolor=blue,
    citecolor=green,
    urlcolor=red]{hyperref}

% ==================== 参考文献 ====================
\usepackage[style=science,backend=biber,sorting=nyt]{biblatex}
\addbibresource{references/references.bib}

% ==================== 定理环境 ====================
\newtheorem{theorem}{Theorem}[section]
\newtheorem{lemma}[theorem]{Lemma}
\newtheorem{proposition}[theorem]{Proposition}
\newtheorem{corollary}[theorem]{Corollary}
\theoremstyle{definition}
\newtheorem{definition}[theorem]{Definition}
\newtheorem{assumption}[theorem]{Assumption}
\theoremstyle{remark}
\newtheorem{remark}[theorem]{Remark}
\newtheorem{example}[theorem]{Example}

% =================一些包的设置=====================
\newcommand{\unsure}[1]{\todo[inline,linecolor=red,backgroundcolor=red!20,bordercolor=red]{#1}}
% 蓝色 - 用于需要修改的内容
\newcommand{\change}[1]{\todo[inline,linecolor=blue,backgroundcolor=blue!20,bordercolor=blue]{#1}}
% 绿色 - 用于补充信息、说明
\newcommand{\info}[1]{\todo[inline,linecolor=green,backgroundcolor=green!20,bordercolor=green]{#1}}
% 橙色 - 用于改进建议
\newcommand{\improvement}[1]{\todo[inline,linecolor=orange,backgroundcolor=orange!20,bordercolor=orange]{#1}}
% 紫色 - 用于疑问
\newcommand{\question}[1]{\todo[inline,linecolor=purple,backgroundcolor=purple!20,bordercolor=purple]{#1}}
% 灰色 - 用于完成状态
\newcommand{\done}[1]{\todo[inline,linecolor=gray,backgroundcolor=gray!20,bordercolor=gray]{#1}}
