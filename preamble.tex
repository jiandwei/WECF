% ==================== 基础包 ====================
\usepackage{amsmath,amsfonts,amssymb,amsthm}
\usepackage{mathtools}
\usepackage[colorinlistoftodos]{todonotes}
% \usepackage{fontspec} % LuaLaTeX 字体支持
\usepackage{ctex}
\usepackage{algorithm}
\usepackage{algorithmicx}
\usepackage{algpseudocode}
% ==================== 页面设置 ====================
\usepackage[margin=2cm]{geometry}
\usepackage{setspace}
\onehalfspacing

% ==================== 字体设置 ====================
% 英文字体
\setmainfont{Times New Roman}
% % 中文字体 - Noto Serif SC(思源宋体,LuaLaTeX 支持可变字体)
% \newfontfamily\CJKmainfont{Noto Serif SC}
% \newfontfamily\CJKsansfont{Noto Sans SC}
% \newfontfamily\CJKmonofont{Noto Sans SC}
% % 设置默认 CJK 字体
% \setmainfont{Noto Serif SC}[Script=CJK]

% ==================== 图表支持 ====================
\usepackage{graphicx}
\usepackage{float}
\usepackage{caption}
\usepackage{subcaption}
\usepackage{booktabs}
\usepackage{multirow}
\usepackage{array}

% ==================== 颜色和链接 ====================
\usepackage{xcolor}
\usepackage[colorlinks=true,
    linkcolor=blue,
    citecolor=green,
    urlcolor=red]{hyperref}

% 智能交叉引用(cleveref)
\usepackage[nameinlink]{cleveref}

% cleveref 的中文名称映射(用于 \cref/\Cref)
\crefname{theorem}{定理}{定理}
\Crefname{theorem}{定理}{定理}
\crefname{lemma}{引理}{引理}
\Crefname{lemma}{引理}{引理}
\crefname{proposition}{命题}{命题}
\crefname{corollary}{推论}{推论}
\crefname{definition}{定义}{定义}
\crefname{assumption}{假设}{假设}
\crefname{remark}{注记}{注记}
\crefname{example}{示例}{示例}

% ==================== 参考文献 ====================
\usepackage[style=science,backend=biber,sorting=nyt]{biblatex}
\addbibresource{references/references.bib}

% ==================== 定理环境(中文化) ====================
\newtheorem{theorem}{定理}[section]
\newtheorem{lemma}{引理}[subsection]
\newtheorem{proposition}[theorem]{命题}
\newtheorem{corollary}[theorem]{推论}
\theoremstyle{definition}
\newtheorem{definition}[theorem]{定义}
\newtheorem{assumption}[theorem]{假设}
\theoremstyle{remark}
\newtheorem{remark}{注记}
\newtheorem{example}[theorem]{示例}

% 证明标题本地化
\renewcommand{\proofname}{证明}

% =================一些包的设置(中文 todonote)=====================
\newcommand{\unsure}[1]{\todo[inline,linecolor=red,backgroundcolor=red!20,bordercolor=red]{\textbf{[待确认]} #1}}
% 蓝色 - 用于需要修改的内容
\newcommand{\change}[1]{\todo[inline,linecolor=blue,backgroundcolor=blue!20,bordercolor=blue]{\textbf{[需修改]} #1}}
% 绿色 - 用于补充信息、说明
\newcommand{\info}[1]{\todo[inline,linecolor=green,backgroundcolor=green!20,bordercolor=green]{\textbf{[补充说明]} #1}}
% 橙色 - 用于改进建议
\newcommand{\improvement}[1]{\todo[inline,linecolor=orange,backgroundcolor=orange!20,bordercolor=orange]{\textbf{[改进建议]} #1}}
% 紫色 - 用于疑问
\newcommand{\question}[1]{\todo[inline,linecolor=purple,backgroundcolor=purple!20,bordercolor=purple]{\textbf{[疑问]} #1}}
% 灰色 - 用于完成状态
\newcommand{\done}[1]{\todo[inline,linecolor=gray,backgroundcolor=gray!20,bordercolor=gray]{\textbf{[已完成]} #1}}
