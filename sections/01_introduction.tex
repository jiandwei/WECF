\section{Introduction}
  \label{sec:introduction}


  Strtuctural breaks are a common phenomenon in economics and finance,often resulting from policy changes,economics crisis,pandemics or techological innovations, even seasonal effects. An obvious political change is Brexit,which makes a great shock to the UK exchange market and even global market \cite{hassan2024global}. When Brexit ended formally, a global pandemic COVID-19 broke out, and a lot of countries implemented lockdown policies to prevent the spread of the virus, which induces unemployment rates to soar and economics growth to stall\cite{asare2021impact}. But when we recover from pandemics and Brexit, China and US are leading the new trechnology revolution with AI and green energy. In particular, China has made significant contributions to the stability of global supply chains and demonstrated steady growth\cite{free2021global}.Besides,preference changes, the war can also leade to strtuctural changes. Many Many empirical studies have documented  the structural instability of economic time series data \cite{stock_evidence_1996,hansen_new_2001,zhang_new_2008,rossi_conditional_2013}.