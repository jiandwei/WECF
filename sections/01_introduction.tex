\section{Introduction}
  \label{sec:introduction}


  Strtuctural breaks are a common phenomenon in economics and finance,often resulting from policy changes,economics crisis,pandemics or techological innovations, even seasonal effects. An obvious political change is Brexit,which makes a great shock to the UK exchange market and even global market \cite{hassan2024global}. When Brexit ended formally, a global pandemic COVID-19 broke out, and a lot of countries implemented lockdown policies to prevent the spread of the virus, which induces unemployment rates to soar and economics growth to stall\cite{asare2021impact}. But when we recover from pandemics and Brexit, China and US are leading the new trechnology revolution with AI and green energy. In particular, China has made significant contributions to the stability of global supply chains and demonstrated steady growth\cite{free2021global}.Besides,preference changes, the war can also leade to strtuctural changes. Many Many empirical studies have documented  the structural instability of economic time series data \cite{stock_evidence_1996,hansen_new_2001,zhang_new_2008,rossi_conditional_2013}.

  Structural breaks in time series have drawn considerable attention. Both econometricians and statisticians have devoted a vast amount of effort to this field. Most works pay more attention on dtecting breaks in discrete time series,such as the linear regression models in eculidean space. Examples include \cite{bai_testing_1996} test for parameter constancy in linear regression model  via the empirical distribution function (EDF),\cite{bai_estimating_1998} approach for multiple structural  breaks in linear regression models. Besides, some people also focus on distribution changes in discrete data. Among them,\cite{cowell_measures_1985} introduces a system of axioms to measure distributional changes. \cite{dumbgen_asymptotic_1991} estimates a structrual break in distribution and develops the corresponding aympototics theorey for independent random variables. \cite{zou_nonparametric_2014} propose a nonparametric maximum likelihood approach for multiple structural breaks based on independent data. \cite{fu_multiple_2023} proposed an estimation and test for multiple structural breaks in distribution via an empirical characteristic function approach. The methods mentioned above are mainly for discrete time series, and with the era of big data and the development of techonology, we could gather more and more complex data, even in seconds. Such data could be viewed as functional data, which are infinite dimensional and lie in a function space. Examples include And these data are not handled well by those discrete time series methods. Therefore, it is of great importance to develop structural break detection methods for functional data. The CUSUM test based on PCA is proposed by \cite{berkes_detecting_2009}, which needs to predefined the number of principal components under independent functional data. \cite{hormann2010weakly,aue_detecting_2018} extend such method to weak dependent, where they pay more glance on the mean function breaks.
  \info{Need to be rewriten again.}

  Thus, we need to develop more general and powerful methods to detect structrual breaks in functional time series. \cite{fu_multiple_2023} uses empirical characteristic function(ECF) to detect breaks in distribution with finite dimensional data, which lakcs the ability to handle functional data. However,\cite{boniece_changepoint_2023} proposes energy method to detect breaks in functional data that assumes the generalized convariance function matains the same before and after the break point. Besides, the method is just to tell us that the distribution are diffrent before and after the break point, but cannot tell us which aspect of the distribution has changed, such as mean, variance, or higher order moments. Therefore, it is necessary to develop a more general method to detect structural breaks in functional data that can identify which aspects of the distribution have changed.
  \todo{I need to do more here}