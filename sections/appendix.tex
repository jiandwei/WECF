% \section{完整的理论证明体系}
%   \label{sec:appendix-proof}
\section{第一章:预备知识与假设条件}\label{sec:prelim-hypotheses}

  \subsection{基本设定}\label{ss:basic-setup}
      \textbf{数据结构}: 观测到函数型时间序列
      \(\{X_t: \mathcal{T} \to \mathbb{R}\}_{t=1}^T\),其中
      \(\mathcal{T} = [0,1]\) 是紧致区间。

      \textbf{结构断点模型}:
      \[X_t(s) = \mu_{j(t)}(s) + \sum_{k=1}^{\infty} \xi_{k,t} \psi_{k,j(t)}(s), \quad s \in \mathcal{T}\]
      其中: \begin{itemize}
          \item  \(j(t) = j\) 当
                \(t \in (T_{j-1}, T_j]\),\(j = 1, \ldots, M_0+1\)
          \item \(\{\psi_{k,j}\}_{k=1}^{\infty}\) 是第 \(j\) 个regime的正交基函数
          \item \(\{\xi_{k,t}\}\) 是随机系数序列
      \end{itemize}

  \subsection{假设条件体系}\label{ss:assumptions}

      \begin{assumption}{$\alpha$-混合性}
          存在常数 \(C > 0\) 和\label{ass:alpha-mixing}
          \(\rho \in (0,1)\) 使得α-混合系数满足
          \[\alpha(n) \leq C \rho^n, \quad \forall n \geq 1\]
      \end{assumption}
      \info{
      \textbf{证明验证方法}: 对于满足如下结构的过程可验证A.1:
      \[X_t(s) = \sum_{l=0}^{\infty} \Psi_l(s, \varepsilon_{t-l})\] 其中
      \(\{\varepsilon_t\}\) 是i.i.d.噪声,\(\Psi_l\) 满足
      \(\sum_{l=0}^{\infty} \|\Psi_l\|_{\infty} < \infty\)。}

      \begin{lemma}{验证混合性}\label{lem:verify-mixing}
          若 \(X_t\) 可表示为上述Bernoulli
          shift形式,则
          \(\alpha(n) \leq C \sum_{l=n}^{\infty} \|\Psi_l\|_{\infty}\)。
      \end{lemma}

      \begin{proof}
          \begin{align*}
              \alpha(n) & = \sup_{A \in \mathcal{F}_{-\infty}^0, B \in \mathcal{F}_n^{\infty}} |P(A \cap B) - P(A)P(B)|                         \\
                        & \leq \sup_{f,g: \|f\|_{\infty}, \|g\|_{\infty} \leq 1} |E[f(X_0, X_{-1}, \ldots) g(X_n, X_{n+1}, \ldots)] - E[f]E[g]| \\
                        & \leq E\left[\sup_f |f(X_0, \ldots) - f(X_0^*, \ldots)|\right]
          \end{align*} 其中 \(X_t^*\) 是用独立噪声 \(\varepsilon_t^*\) 替换
          \(\varepsilon_t\)(\(t \geq n\))构造的。由于
          \[\|X_n - X_n^*\|_{\infty} \leq \sum_{l=n}^{\infty} \|\Psi_l\|_{\infty} \cdot \|\varepsilon_0 - \varepsilon_0^*\|_{\infty}\]
          利用耦合不等式得证。
      \end{proof}

      \begin{assumption}{矩条件}
          \[E\left[\left(\int_{\mathcal{T}} X_t^2(s)ds\right)^{2+\delta}\right] < \infty\]\label{ass:moment}
          对某个 \(\delta > 0\) 成立。
      \end{assumption}

      \begin{assumption}{断点大小}
          对所有\label{ass:break-size}
          \(j = 1, \ldots, M_0\),断点大小满足
          \[\Delta_j = \inf_{u \in U} \|\phi_{j+1}(u) - \phi_j(u)\|_{L^2(W)} > 0\]
          其中
          \(\phi_j(u) = E[e^{iu\int X_t(s)ds} \mid t \in \text{regime } j]\)。
      \end{assumption}

      \begin{assumption}{协方差结构的分段平稳性}
          在每个regime内,广义误差函数\label{ass:cov-piecewise}
          \(\varepsilon_t(u) = e^{iu\int X_t(s)ds} - \phi_j(u)\) 的长期协方差核
          \[\Omega_j(u,v) = \sum_{l=-\infty}^{\infty} \text{Cov}[\varepsilon_0^{(j)}(u), \varepsilon_l^{(j)}(v)]\]
          存在且满足:
          \begin{enumerate}
              \item \(\Omega_j(u,v)\) 作为 \(L^2(U \times U, W \otimes W)\)的算子是正定的
              \item \(\sup_j \|\Omega_j\|_{\text{op}} < \infty\)
              \item 不同regime的协方差核可以不同,即\(\Omega_j \neq \Omega_k\) for \(j \neq k\)
                    \info{这个我们与Boniece的关键区别}
          \end{enumerate}
      \end{assumption}

      \begin{remark}
          这与Boniece的Assumption A.4不同,他们假设全局\(\Omega(u,v)\) 不变。我们的设定更现实但理论更复杂。
      \end{remark}

      \begin{assumption}{权重函数}
          权重函数\(W: \mathbb{R} \to \mathbb{R}_+\) 满足:\label{ass:weight-function}

          \begin{enumerate}
              \item \(W(u)\)是连续、对称的概率密度
              \item \(\int_{\mathbb{R}} |u|^4 W(u)du < \infty\)
              \item \(W(u)\) 的特征函数 \(\hat{W}(t) = \int e^{itu}W(u)du\) 在\(\mathbb{R}\) 上平方可积
          \end{enumerate}
      \end{assumption}

      \begin{remark}
          \textbf{典型选择}:\(W(u) = (2\pi)^{-1/2}e^{-u^2/2}\)(标准正态密度)
      \end{remark}

      \begin{assumption}{trimming参数}
          存在\(\varepsilon \in (0, 1/4)\) 使得对所有 \(j\),\[\min\{r_j^0 - r_{j-1}^0\} \geq 2\varepsilon\]\label{ass:trimming}
      \end{assumption}

      \begin{assumption}{算子的谱间隙}
          协方差算子\(C_j: L^2(\mathcal{T}) \to L^2(\mathcal{T})\) 定义为\[(C_j f)(s) = \int_{\mathcal{T}} \text{Cov}[X_t^{(j)}(s), X_t^{(j)}(t)] f(t)dt\]的特征值 \(\lambda_{1,j} \geq \lambda_{2,j} \geq \cdots\) 满足\[\inf_j (\lambda_{k,j} - \lambda_{k+1,j}) > 0, \quad \forall k \geq 1\]\label{ass:spectral-gap}
      \end{assumption}

\section{第二章:泛函中心极限定理的分段版本}\label{sec:fclt}

  \subsection{单个regime内的FCLT}\label{ss:fclt-single}

      \begin{theorem}{(Regime内的函数型CLT)} 在\cref{ass:alpha-mixing,ass:moment}下,对固定的regime \(j\),定义\label{thm:regime-clt}
          \[S_{n,j}(r, u) = \frac{1}{\sqrt{n}} \sum_{i=1}^{[\lfloor nr \rfloor]} \varepsilon_i^{(j)}(u)\]
          其中
          \(\varepsilon_i^{(j)}(u) = e^{iu\int X_{T_{j-1}+i}(s)ds} - \phi_j(u)\)。

          则在 \(C[0,1] \times L^2(U, W)\) 上,有\(S_{n,j}(\cdot, \cdot) \Rightarrow B_j(\cdot, \cdot)\) 其中\(B_j(r, u)\) 是均值为零的Gaussian过程,协方差函数为\[E[B_j(r, u)B_j(s, v)] = \min\{r, s\ \cdot \Omega_j(u, v)\}\]
      \end{theorem}
      \begin{proof}
          \textbf{Step 1:有限维分布收敛}

          固定 \((r_1, \ldots, r_k) \in [0,1]^k\) 和
          \((u_1, \ldots, u_m) \in U^m\)。定义向量
          \[Z_n = (S_{n,j}(r_1, u_1), S_{n,j}(r_1, u_2), \ldots, S_{n,j}(r_k, u_m))^T \in \mathbb{R}^{km}\]

          我们需证明 \(Z_n \xrightarrow{d} N(0, \Sigma)\),其中
          \[\Sigma_{(i,l),(i',l')} = \min\{r_i, r_{i'}\} \cdot \Omega_j(u_l, u_{l'})\]


          \begin{lemma}{混合序列的CLT)}
              \label{mixing-clt}
              (Ibragimov \& Linnik, 1971) 若
              \(\{Y_i\}\) 是均值为零的强混合序列,满足
              \[\sum_{n=1}^{\infty} \alpha^{\delta/(2+\delta)}(n) < \infty\] 且
              \(E|Y_1|^{2+\delta} < \infty\),则
              \[\frac{1}{\sqrt{n}} \sum_{i=1}^n Y_i \xrightarrow{d} N(0, \sigma^2)\]
              其中 \(\sigma^2 = \sum_{l=-\infty}^{\infty} \text{Cov}(Y_0, Y_l)\)。
          \end{lemma}

          应用\cref{mixing-clt}到每个分量:对线性组合
          \[\sum_{i=1}^{km} a_i Z_n^{(i)} = \frac{1}{\sqrt{n}} \sum_{t=1}^{[\lfloor n r_{\max} \rfloor]} \left[\sum_{i,l} a_{(i,l)} \mathbb{1}\{t \leq [\lfloor nr_i \rfloor]\} \varepsilon_t^{(j)}(u_l)\right]\]

          由Cramér-Wold定理,只需证明每个线性组合收敛到正态分布。由于
          \(\{\varepsilon_t^{(j)}(u)\}\) 满足\cref{ass:alpha-mixing}的混合性(继承自 \(\{X_t\}\)),且
          \[E|\varepsilon_1^{(j)}(u)|^{2+\delta} \leq E\left[|e^{iu\int X_1(s)ds}|^{2+\delta}\right] + |\phi_j(u)|^{2+\delta} \leq 1 + 1 < \infty\]

          应用\cref{mixing-clt}得有限维收敛。

          \textbf{Step 2:紧性(Tightness)}

          需证明对任意 \(\eta > 0\),存在紧集
          \(K_{\eta} \subset C[0,1] \times L^2(U, W)\) 使得
          \[\limsup_{n \to \infty} P(S_{n,j} \notin K_{\eta}) < \eta\]
          \begin{lemma}{函数空间的紧性判据}
              \info{(Bosq, 2000, Theorem 7.1),可以从这里得到}
              \label{functional-space-tightness}
              若对任意 \(\epsilon > 0\) 和 \(\eta > 0\),存在 \(\delta > 0\) 使得
              \[\limsup_{n \to \infty} P\left(\sup_{\substack{|r-s|<\delta \\ \|u-v\|<\delta}} |S_{n,j}(r,u) - S_{n,j}(s,v)| > \epsilon\right) < \eta\]
              则 \(\{S_{n,j}\}\) 是紧的。
          \end{lemma}

          \textbf{控制模量}: \begin{align*}
              E|S_{n,j}(r,u) - S_{n,j}(s,v)|^2 & = \frac{1}{n} E\left|\sum_{i=[\lfloor ns \rfloor]+1}^{[\lfloor nr \rfloor]} \varepsilon_i^{(j)}(u) + \sum_{i=1}^{[\lfloor ns \rfloor]} [\varepsilon_i^{(j)}(u) - \varepsilon_i^{(j)}(v)]\right|^2 \\
                                               & \leq \frac{1}{n} \cdot n|r-s| \cdot \|\Omega_j\|_{\text{op}} + \frac{1}{n} \cdot ns \cdot E|\varepsilon_1^{(j)}(u) - \varepsilon_1^{(j)}(v)|^2                                                    \\
                                               & \leq C(|r-s| + \|u-v\|)
          \end{align*}

          其中第二项利用了
          \[|\varepsilon_1^{(j)}(u) - \varepsilon_1^{(j)}(v)| = |e^{iu\int X_1(s)ds} - e^{iv\int X_1(s)ds}| \leq |u-v| \cdot \left|\int X_1(s)ds\right|\]

          由Chebyshev不等式,
          \[P\left(|S_{n,j}(r,u) - S_{n,j}(s,v)| > \epsilon\right) \leq \frac{C(|r-s| + \|u-v\|)}{\epsilon^2}\]

          选择 \(\delta = \epsilon^3/C\),则右边
          \(< \epsilon\)。应用\cref{functional-space-tightness}得紧性。

          \textbf{Step 3:唯一性}

          极限过程的唯一性由协方差结构唯一确定(Gaussian过程的特征)。
      \end{proof}
  \subsection{跨regime的联合弱收敛
      }\label{ss:fclt-cross}

      \info{这是我们方法的核心创新!}

      \begin{theorem}{(全局分段FCLT)} 在\cref{ass:alpha-mixing,ass:moment,ass:break-size,ass:cov-piecewise}下,定义全样本的部分和过程\label{thm:global-fclt}
          \[S_T(r, u) = \frac{1}{\sqrt{T}} \sum_{t=1}^{[\lfloor Tr \rfloor]} \varepsilon_t(u)\]
          其中 \(\varepsilon_t(u) = e^{iu\int X_t(s)ds} - \phi_{j(t)}(u)\)。

          则在 \(C[0,1] \times L^2(U, W)\) 上,
          \[S_T(\cdot, \cdot) \Rightarrow \mathcal{B}(\cdot, \cdot)\] 其中
          \(\mathcal{B}(r, u)\) 是\textbf{分段}Gaussian过程:
          \[\mathcal{B}(r, u) = \sum_{j=1}^{M_0+1} \mathbb{1}\{r_{j-1}^0 < r \leq r_j^0\} \left[B_j(r - r_{j-1}^0, u) + B_j(r_{j-1}^0, u)\right]\]
          其中 \(\{B_j\}_{j=1}^{M_0+1}\)
          是\textbf{独立}的Gaussian过程,每个满足\cref{thm:regime-clt}。

          \textbf{关键性质}:
          \begin{enumerate}
              \item  \(\mathcal{B}(r, u)\) 在 \(r = r_j^0\)处\textbf{连续}(路径连续)
              \item 但在 \(r_j^0\)
                    处\textbf{不可微}(因为协方差核 \(\Omega_j\) 变化)
              \item  协方差函数为
                    \[E[\mathcal{B}(r, u)\mathcal{B}(s, v)] = \sum_{j=1}^{M_0+1} \min\{(r - r_{j-1}^0)^+, (s - r_{j-1}^0)^+\} \cdot \Omega_j(u, v) \cdot \mathbb{1}\{r, s \in (r_{j-1}^0, r_j^0]\}\]
          \end{enumerate}
      \end{theorem}

      \begin{proof}
          \textbf{Step 1:分段分解}

          将 \(S_T(r, u)\) 分解为各regime的贡献:
          \[S_T(r, u) = \sum_{j=1}^{M_0+1} \frac{1}{\sqrt{T}} \sum_{t=T_{j-1}^0+1}^{\min(T_j^0, [\lfloor Tr \rfloor])} \varepsilon_t(u)\]

          \begin{lemma}{(Regime间的独立性)}\label{lem:regime-independence}
              对 \(j \neq k\),随机过程
              \[S_{T,j}(r, u) = \frac{1}{\sqrt{T}} \sum_{t=T_{j-1}^0+1}^{T_j^0} \varepsilon_t(u), \quad S_{T,k}(r, u) = \frac{1}{\sqrt{T}} \sum_{t=T_{k-1}^0+1}^{T_k^0} \varepsilon_t(u)\]
              是\textbf{渐近独立}的。
          \end{lemma}

          \begin{proof}
              利用混合性,对任意有界连续函数 \(f, g\), \begin{align*}
                  |E[f(S_{T,j}) g(S_{T,k})] - E[f(S_{T,j})] E[g(S_{T,k})]| & \leq \|f\|_{\infty} \|g\|_{\infty} \cdot \alpha(T_{k-1}^0 - T_j^0) \\
                                                                           & \leq C \rho^{T(r_{k-1}^0 - r_j^0)} \to 0
              \end{align*} 当 \(T \to \infty\) 时。
          \end{proof}

          \textbf{Step 2:标准化变换}

          对每个regime \(j\),定义标准化过程
          \[\tilde{S}_{T,j}(r, u) = \frac{1}{\sqrt{T(r_j^0 - r_{j-1}^0)}} \sum_{t=T_{j-1}^0+1}^{T_{j-1}^0 + [\lfloor T(r_j^0 - r_{j-1}^0) r \rfloor]} \varepsilon_t(u)\]

          由\cref{thm:regime-clt},
          \[\tilde{S}_{T,j}(\cdot, \cdot) \Rightarrow B_j(\cdot, \cdot)\]

          \textbf{Step 3:拼接各段}

          全局过程可表示为
          \[S_T(r, u) = \sqrt{T} \sum_{j: r_j^0 < r} (r_j^0 - r_{j-1}^0) \tilde{S}_{T,j}(1, u) + \sqrt{T}(r - r_{j(r)-1}^0) \tilde{S}_{T,j(r)}\left(\frac{r - r_{j(r)-1}^0}{r_{j(r)}^0 - r_{j(r)-1}^0}, u\right)\]
          其中 \(j(r) = \max\{j: r_j^0 < r\}\)。

          \begin{lemma}{(路径连续性)} 在 \(r = r_j^0\) 处,\label{lem:path-continuity}
              \[\lim_{r \uparrow r_j^0} S_T(r, u) = \lim_{r \downarrow r_j^0} S_T(r, u) \quad \text{in probability}\]
          \end{lemma}

          \begin{proof}
              \begin{align*}
                   & S_T(r_j^0 + h, u) - S_T(r_j^0, u)                                                                                       \\
                   & = \frac{1}{\sqrt{T}} \sum_{t=T_j^0+1}^{T_j^0 + [\lfloor Th \rfloor]} \varepsilon_t(u)                                   \\
                   & = \sqrt{h} \cdot \frac{1}{\sqrt{[\lfloor Th \rfloor]}} \sum_{t=T_j^0+1}^{T_j^0 + [\lfloor Th \rfloor]} \varepsilon_t(u) \\
                   & = \sqrt{h} \cdot O_P(1) \to 0
              \end{align*} 当 \(h \to 0\) 时。
          \end{proof}

          \textbf{Step 4:联合弱收敛}

          由Skorohod表示定理(Billingsley, 1968,Theorem 6.7),存在概率空间使得
          \[\sup_{r \in [0,1], u \in U} |S_T(r, u) - \mathcal{B}(r, u)| \to 0 \quad a.s.\]因此在原空间上有弱收敛。
      \end{proof}

  \subsection{Gaussian
          Bridge的构造}\label{ss:gaussian-bridge}
      \begin{definition}\label{def:gaussian-bridge}
          定义分段Gaussian bridge为
          \[\mathcal{B}^0(r, u) = \mathcal{B}(r, u) - r \cdot \mathcal{B}(1, u)\]
      \end{definition}

      \begin{proposition}\label{prop:bridge-properties}
          \(\mathcal{B}^0(r, u)\) 满足:
          \begin{enumerate}
              \item \(\mathcal{B}^0(0, u) = \mathcal{B}^0(1, u) = 0\)
              \item 协方差函数为
                    \[E[\mathcal{B}^0(r, u)\mathcal{B}^0(s, v)] = \sum_{j=1}^{M_0+1} [\min\{r, s\} - rs] \cdot \Omega_j(u, v) \cdot \mathbb{1}\{r, s \in (r_{j-1}^0, r_j^0]\}\]
          \end{enumerate}
      \end{proposition}

      \begin{proof}
          直接计算 \begin{align*}
              E[\mathcal{B}^0(r, u)\mathcal{B}^0(s, v)] & = E[\mathcal{B}(r, u)\mathcal{B}(s, v)] - rE[\mathcal{B}(r, u)\mathcal{B}(1, v)]         \\
                                                        & \quad - sE[\mathcal{B}(1, u)\mathcal{B}(s, v)] + rsE[\mathcal{B}(1, u)\mathcal{B}(1, v)]
          \end{align*}

          当 \(r, s\) 在同一regime内时(如regime \(j\)),
          \[E[\mathcal{B}(r, u)\mathcal{B}(s, v)] = \min\{r - r_{j-1}^0, s - r_{j-1}^0\} \cdot \Omega_j(u, v)\]
          \[E[\mathcal{B}(r, u)\mathcal{B}(1, v)] = (r - r_{j-1}^0) \cdot \Omega_j(u, v)\]

          代入得
          \[[\min\{r - r_{j-1}^0, s - r_{j-1}^0\} - r(r_j^0 - r_{j-1}^0) - s(r_j^0 - r_{j-1}^0) + rs] \cdot \Omega_j(u, v)\]

          在 \((r_{j-1}^0, r_j^0]\) 上归一化后得到标准bridge形式。
      \end{proof}

\section{第三章:断点估计的渐近理论}\label{sec:breakpoint-asymptotics}

  \subsection{目标函数的一致收敛}\label{ss:objective-uniform}

      \begin{definition}{(SSGR的大数定律)}\label{thm:ssgr-lln} 在Assumptions A.1-A.4*下,定义
          \[Q_T(r_1, \ldots, r_M) = \frac{1}{T} \text{SSGR}_M(r_1, \ldots, r_M)\]

          则对任意紧集 \(K \subset [0,1]^M\),
          \[\sup_{(r_1,\ldots,r_M) \in K} |Q_T(r_1, \ldots, r_M) - Q(r_1, \ldots, r_M)| \xrightarrow{a.s.} 0\]
          其中
          \[Q(r_1, \ldots, r_M) = \sum_{j=1}^{M+1} (r_j - r_{j-1}) \int_U \text{Var}[\varepsilon_t^{(j_r)}(u)] W(u)du\]
          \(j_r\) 表示 \((r_{j-1}, r_j]\) 中真实断点所属的regime。
      \end{definition}

      \begin{proof}
          \textbf{Step 1:展开SSGR}

          \begin{align*}
              \text{SSGR}_M & = \sum_{j=1}^{M+1} \sum_{t=T_{j-1}+1}^{T_j} \int_U \left|e^{iu\int X_t(s)ds} - \tilde{\phi}_j(u)\right|^2 W(u)du                                         \\
                            & = \sum_{j=1}^{M+1} \sum_{t=T_{j-1}+1}^{T_j} \int_U \left|\varepsilon_t(u) + [\phi_{j(t)}^0(u) - \tilde{\phi}_j(u)]\right|^2 W(u)du                       \\
                            & = \underbrace{\sum_{j=1}^{M+1} \sum_{t=T_{j-1}+1}^{T_j} \int_U |\varepsilon_t(u)|^2 W(u)du}_{I_1}                                                        \\
                            & \quad + \underbrace{2\sum_{j=1}^{M+1} \sum_{t=T_{j-1}+1}^{T_j} \int_U \text{Re}[\varepsilon_t(u) (\phi_{j(t)}^0(u) - \tilde{\phi}_j(u))^*] W(u)du}_{I_2} \\
                            & \quad + \underbrace{\sum_{j=1}^{M+1} (T_j - T_{j-1}) \int_U |\phi_{j(t)}^0(u) - \tilde{\phi}_j(u)|^2 W(u)du}_{I_3}
          \end{align*}

          \textbf{Step 2:控制 \(I_1\) 项}

          \begin{lemma}\label{lem:segment-lln}
              在每个segment内,
              \[\frac{1}{T_j - T_{j-1}} \sum_{t=T_{j-1}+1}^{T_j} \int_U |\varepsilon_t(u)|^2 W(u)du \xrightarrow{a.s.} \int_U E[|\varepsilon_1^{(j_r)}(u)|^2] W(u)du\]
          \end{lemma}

          \begin{proof}
              由遍历定理(Ergodic Theorem,Krengel,
              1985),对平稳遍历序列
              \[\frac{1}{n} \sum_{i=1}^n g(Y_i) \xrightarrow{a.s.} E[g(Y_1)]\]

              在segment \((T_{j-1}, T_j]\) 内,若该segment完全包含在某个真实regime
              \(j_r\) 中(即不跨越断点),则
              \(\{\varepsilon_t(u)\}_{t=T_{j-1}+1}^{T_j}\)
              是平稳混合的(由A.1),应用遍历定理得
              \[\frac{1}{T_j - T_{j-1}} \sum_{t=T_{j-1}+1}^{T_j} |\varepsilon_t(u)|^2 \xrightarrow{a.s.} E[|\varepsilon_1^{(j_r)}(u)|^2] = \Omega_{j_r}(u, u)\]

              对 \(u\) 积分(由Fubini定理)得证。
          \end{proof}

          \textbf{Step 3:控制 \(I_2\) 项(关键)}

          \begin{lemma}\label{lem:cross-term}
              交叉项满足
              \[\frac{1}{\sqrt{T}} \sum_{t=T_{j-1}+1}^{T_j} \int_U \varepsilon_t(u) (\phi_{j(t)}^0(u) - \tilde{\phi}_j(u))^* W(u)du = o_P(T)\]
          \end{lemma}

          \begin{proof}
              \textbf{情况1}:segment \((T_{j-1}, T_j]\) 完全在某个真实regime内 则
              \(\phi_{j(t)}^0(u) = \phi_{j_r}^0(u)\) 不依赖于 \(t\),且
              \[\tilde{\phi}_j(u) = \frac{1}{T_j - T_{j-1}} \sum_{t=T_{j-1}+1}^{T_j} e^{iu\int X_t(s)ds} \xrightarrow{a.s.} \phi_{j_r}^0(u)\]

              因此 \[\phi_{j(t)}^0(u) - \tilde{\phi}_j(u) = o_P(1)\]

              从而
              \[\left|\sum_{t=T_{j-1}+1}^{T_j} \varepsilon_t(u) (\phi_{j(t)}^0(u) - \tilde{\phi}_j(u))^*\right| \leq \left|\sum_{t=T_{j-1}+1}^{T_j} \varepsilon_t(u)\right| \cdot |\phi_{j_r}^0(u) - \tilde{\phi}_j(u)| = O_P(\sqrt{T_j - T_{j-1}}) \cdot o_P(1) = o_P(\sqrt{T})\]

              \textbf{情况2}:segment \((T_{j-1}, T_j]\) 跨越断点 \(r_k^0\)
              不失一般性,假设 \(T_{j-1} < T_k^0 < T_j\)。则
              \[\tilde{\phi}_j(u) = \lambda_j \phi_k^0(u) + (1-\lambda_j) \phi_{k+1}^0(u)\]
              其中 \(\lambda_j = (T_k^0 - T_{j-1})/(T_j - T_{j-1})\)。

              交叉项可分解为: \begin{align*}
                   & \sum_{t=T_{j-1}+1}^{T_j} \varepsilon_t(u) (\phi_{j(t)}^0(u) - \tilde{\phi}_j(u))^*                                                                                                     \\
                   & = \sum_{t=T_{j-1}+1}^{T_k^0} \varepsilon_t(u) [\phi_k^0(u) - \tilde{\phi}_j(u)]^* + \sum_{t=T_k^0+1}^{T_j} \varepsilon_t(u) [\phi_{k+1}^0(u) - \tilde{\phi}_j(u)]^*                    \\
                   & = \sum_{t=T_{j-1}+1}^{T_k^0} \varepsilon_t(u) [(1-\lambda_j)(\phi_k^0(u) - \phi_{k+1}^0(u))]^* + \sum_{t=T_k^0+1}^{T_j} \varepsilon_t(u) [-\lambda_j(\phi_k^0(u) - \phi_{k+1}^0(u))]^*
              \end{align*}

              注意到
              \(\sum_{t=T_{j-1}+1}^{T_k^0} \varepsilon_t(u) = O_P(\sqrt{T_k^0 - T_{j-1}})\),而
              \[(1-\lambda_j) = \frac{T_j - T_k^0}{T_j - T_{j-1}}\]

              因此该项的贡献为
              \[O_P(\sqrt{T_k^0 - T_{j-1}}) \cdot \frac{T_j - T_k^0}{T_j - T_{j-1}} \cdot \|\Delta_k\| = O_P\left(\frac{\sqrt{T_k^0 - T_{j-1}}(T_j - T_k^0)}{T_j - T_{j-1}}\right)\]

              当 \(T_j - T_{j-1} \asymp T\) 时(由A.6保证),此项为
              \(O_P(\sqrt{T})\),因此
              \[\frac{1}{\sqrt{T}} \cdot O_P(\sqrt{T}) = O_P(1)\]

              在 \(I_2\) 中的贡献为 \(O_P(\sqrt{T})\),相对于 \(I_1\) 和 \(I_3\) 的
              \(O(T)\) 是低阶项。
          \end{proof}



          \textbf{Step 4:分析 \(I_3\) 项(断点惩罚)}

          \begin{lemma}\label{lem:I3-limit}
              \[\frac{1}{T} I_3 \xrightarrow{P} \begin{cases}
                      0                                                               & \text{if } (r_1, \ldots, r_M) = (r_1^0, \ldots, r_{M_0}^0) \\
                      \sum_{j: \text{misalign*ed}} (r_j - r_{j-1}) \|\Delta_{j_r}\|^2 & \text{otherwise}
                  \end{cases}\]
          \end{lemma}

          \begin{proof}
              当某个segment跨越真实断点时,
              \[\int_U |\phi_{j(t)}^0(u) - \tilde{\phi}_j(u)|^2 W(u)du = \int_U |\lambda_j \phi_k^0(u) + (1-\lambda_j)\phi_{k+1}^0(u) - \phi_{j(t)}^0(u)|^2 W(u)du\]

              \begin{itemize}
                  \item
                        若 \(t \in \text{regime } k\):\(\phi_{j(t)}^0 = \phi_k^0\),则
                        \[= (1-\lambda_j)^2 \|\Delta_k\|^2\]
                  \item
                        若 \(t \in \text{regime } k+1\):\(\phi_{j(t)}^0 = \phi_{k+1}^0\),则
                        \[= \lambda_j^2 \|\Delta_k\|^2\]
              \end{itemize}

              因此
              \[I_3 = (T_k^0 - T_{j-1})(1-\lambda_j)^2 \|\Delta_k\|^2 + (T_j - T_k^0)\lambda_j^2 \|\Delta_k\|^2\]

              利用 \(\lambda_j = (T_k^0 - T_{j-1})/(T_j - T_{j-1})\),可化简为
              \[I_3 = \frac{(T_k^0 - T_{j-1})(T_j - T_k^0)}{T_j - T_{j-1}} \|\Delta_k\|^2\]

              除以 \(T\) 得
              \[\frac{I_3}{T} = \frac{(r_k^0 - r_{j-1})(r_j - r_k^0)}{r_j - r_{j-1}} \|\Delta_k\|^2 + o(1)\]

              这是严格正的(由A.3)。
          \end{proof}
          \question{这个为何可以得到}

          \textbf{Step 5:一致收敛性}

          结合引理\cref{lem:segment-lln,lem:cross-term,lem:I3-limit},我们得到
          \[Q_T(r_1, \ldots, r_M) = \frac{1}{T}[I_1 + I_2 + I_3] \xrightarrow{a.s.} Q(r_1, \ldots, r_M)\]

          一致性利用\textbf{随机等度连续性}(Stochastic Equicontinuity):

          \begin{lemma}
              对任意 \(\epsilon > 0\),存在 \(\delta > 0\) 使得
              \[\limsup_{T \to \infty} P\left(\sup_{\|(r,r')\| < \delta} |Q_T(r) - Q_T(r')| > \epsilon\right) = 0\]
          \end{lemma}

          \begin{proof}
              利用
              \[|Q_T(r) - Q_T(r')| \leq \sum_{j=1}^M |r_j - r_j'| \cdot \sup_t \int_U |\varepsilon_t(u)|^2 W(u)du\]

              由\cref{ass:moment},\(\sup_t E[\int |\varepsilon_t(u)|^2 W(u)du] < \infty\),应用Markov不等式得证。
          \end{proof}

          由Arzelà-Ascoli定理的随机版本(参考van der Vaart \& Wellner, 1996,Theorem 1.5.7),得 \(\{Q_T\}\)
          的\textbf{渐近紧性},结合逐点收敛得一致收敛。
      \end{proof}



  \subsection{断点估计的一致性}\label{ss:consistency}

      \begin{theorem}{(强一致性)} 在\cref{ass:alpha-mixing,ass:moment,ass:break-size,ass:cov-piecewise,ass:weight-function,ass:trimming}下,断点估计量
          \(\hat{r}_j\) 满足
          \[\hat{r}_j \xrightarrow{a.s.} r_j^0, \quad j = 1, \ldots, M_0\]
      \end{theorem}
      \begin{proof}
          \cref{thm:ssgr-lln}
          \[\hat{r} = \arg\min_{r \in \Lambda_\varepsilon} Q_T(r) \xrightarrow{a.s.} \arg\min_{r \in \Lambda_\varepsilon} Q(r) = r^0\]
      \end{proof}

      \info{此外,为了证明在这里取得唯一的极值是唯一的一个一,需验证 \(Q(r)\) 在 \(r^0\) 处有唯一极小值。}

      \begin{lemma}{(目标函数的可识别性)} 对任意
          \(r \neq r^0\),\(Q(r) > Q(r^0)\)
      \end{lemma}

      \begin{proof}
          \[Q(r) - Q(r^0) = \sum_{j: \text{misalign*ed}} (r_j - r_{j-1}) \int_U \text{Var}[\varepsilon^{(\text{mixed})}(u)] W(u)du - \sum_{j} (r_j^0 - r_{j-1}^0) \int_U \text{Var}[\varepsilon^{(j)}(u)] W(u)du\]

          \textbf{关键观察:}当segment包含来自不同regime的数据时,
          \[\text{Var}[\varepsilon^{(\text{mixed})}(u)] = \text{Var}[\varepsilon^{(j)}(u)] + \lambda(1-\lambda)\|\Delta_j\|^2\]
          (这是混合分布方差分解的结果)
          因此
          \(Q(r) - Q(r^0) \geq C \sum_j |r_j - r_j^0| \|\Delta_j\|^2 > 0\)(由A.3)。

          应用\textbf{Argmin连续性定理}(Kim \& Pollard, 1990, Theorem
          2.7)完成证明。
      \end{proof}

  \subsection{断点估计的收敛速度}\label{ss:rate}

      \begin{theorem}{(Rate-T收敛)}
          \label{thm:rate-t-convergence}
          对任意 \(j = 1, \ldots, M_0\),
          \[T(\hat{r}_j - r_j^0) = O_P(1)\]
      \end{theorem}

      \begin{proof}
          \textbf{Step 1:目标函数的局部二次近似}

          在 \(r_j^0\) 的邻域 \(|h| < C/T\) 内,定义
          \[q_T(h) = Q_T(r_1^0, \ldots, r_{j-1}^0, r_j^0 + h/T, r_{j+1}^0, \ldots, r_M^0)\]

          \begin{lemma}{(局部泰勒展开)}
              \label{local-tylor-expansion}
              \[q_T(h) - q_T(0) = -\frac{2h}{\sqrt{T}} Z_T + \frac{h^2}{T} \Delta_j^2 + R_T(h)\]
              其中:
              \begin{itemize}
                  \item \(Z_T = \int_U \text{Re}\left[\frac{1}{\sqrt{T}}\sum_{t=T_j^0-[|h|]}^{T_j^0+[|h|]} \varepsilon_t(u) \Delta_j^*(u)\right] W(u)du = O_P(1)\)
                  \item \(\Delta_j^2 = \int_U |\phi_{j+1}^0(u) - \phi_j^0(u)|^2 W(u)du > 0\)(by A.3)
                  \item \(R_T(h) = o_P(h^2/T)\) 一致地在 \(|h| < C\) 上
              \end{itemize}
          \end{lemma}
          \begin{proof}

              将断点从 \(r_j^0\) 移动到 \(r_j^0 + h/T\)
              后,SSGR的变化来自两个segment:

              \textbf{Segment \(j\)}:从 \((T_{j-1}^0, T_j^0]\) 变为
              \((T_{j-1}^0, T_j^0 + [h]]\)
              \[\Delta_j^{(\text{seg})} = \sum_{t=T_j^0+1}^{T_j^0+[h]} \int_U \left|e^{iu\int X_t(s)ds} - \tilde{\phi}_j^{\text{new}}(u)\right|^2 W(u)du\]

              在新的分割下,
              \[\tilde{\phi}_j^{\text{new}}(u) \approx \phi_j^0(u) + \frac{[h]}{T_j^0 - T_{j-1}^0 + [h]} [\phi_{j+1}^0(u) - \phi_j^0(u)]\]

              对 \(t \in (T_j^0, T_j^0 + [h]]\)(这些点原本属于regime \(j+1\)),
              \[e^{iu\int X_t(s)ds} - \tilde{\phi}_j^{\text{new}}(u) \approx \varepsilon_t^{(j+1)}(u) + \phi_{j+1}^0(u) - \phi_j^0(u) - \frac{[h]}{T_j^0 - T_{j-1}^0} \Delta_j(u)\]

              展开平方并取期望: \begin{align*}
                  E[\Delta_j^{(\text{seg})}] & \approx [h] \int_U \left\{\Omega_{j+1}(u,u) + \left(1 - \frac{[h]}{T_j^0 - T_{j-1}^0}\right)^2 |\Delta_j(u)|^2\right\} W(u)du \\
                                             & \approx [h] \int_U \Omega_{j+1}(u,u) W(u)du + [h] \int_U |\Delta_j(u)|^2 W(u)du + O\left(\frac{[h]^2}{T}\right)               \\
                                             & = [h] \cdot \left[\int_U \Omega_{j+1}(u,u) W(u)du + \Delta_j^2\right] + O(h^2/T)
              \end{align*}

              类似地,\textbf{Segment \(j+1\)} 失去了这 \([h]\) 个点,贡献为
              \[-[h] \cdot \int_U \Omega_{j+1}(u,u) W(u)du\]

              两者相加得净贡献: \[[h] \cdot \Delta_j^2 + O(h^2/T)\]

              围绕均值的波动部分为
              \[\sum_{t=T_j^0+1}^{T_j^0+[h]} \varepsilon_t^{(j+1)}(u) \cdot \Delta_j^*(u) = O_P(\sqrt{[h]})\]

              因此
              \[q_T(h) - q_T(0) = \frac{[h]}{T} \Delta_j^2 + \frac{2}{\sqrt{T}} \sum_{t=T_j^0+1}^{T_j^0+[h]} \text{Re}[\varepsilon_t^{(j+1)}(u) \Delta_j^*(u)] + O(h^2/T)\]

              将 \([h]\) 替换为 \(h\) 并定义 \(Z_T\) 得证。
          \end{proof}

          \textbf{Step 2:应用凸性论证}

          由\cref{local-tylor-expansion},\(q_T(h)\) 在 \(h = 0\) 附近可近似为
          \[q_T(h) \approx q_T(0) + \frac{h^2}{T} \Delta_j^2 + \frac{-2h}{\sqrt{T}} Z_T\]

          最小值点满足
          \[\hat{h} = \arg\min_h q_T(h) \approx \frac{Z_T}{\Delta_j^2 \sqrt{T}}\]

          因此
          \[T(\hat{r}_j - r_j^0) = T \cdot \frac{\hat{h}}{T} = \hat{h} \approx \frac{Z_T}{\Delta_j^2} = O_P(1)\]

          \begin{remark}
              \textbf{严格证明}需验证:
              \begin{enumerate}
                  \item \(q_T(h)\) 在 \(|h| < C\)
                        上是\textbf{近似凸}的(二阶导数有界)
                  \item \(\hat{h}\)
                        确实在此区域内(紧性论证)
              \end{enumerate}
          \end{remark}

          \begin{lemma}{(紧性)} 对任意 \(M > 0\),
              \[P\left(|\hat{h}| > M\right) \leq P\left(q_T(M) < q_T(0)\right) + P\left(q_T(-M) < q_T(0)\right)\]
          \end{lemma}

          由\cref{local-tylor-expansion},当 \(M \to \infty\) 时,右边趋于
          \[P\left(\frac{M^2}{T} \Delta_j^2 - \frac{2M}{\sqrt{T}} Z_T < 0\right) \to P(Z_T > \infty) = 0\]
          因此 \(T(\hat{r}_j - r_j^0) = O_P(1)\)。
      \end{proof}

      \todo{做到了这里,明天再继续写}
\section{第四章:加权统计量的极限理论}\label{sec:weighted-statistics}

  \subsection{零假设下的弱收敛}\label{ss:weighted-null}
      \begin{theorem}{(加权supF的极限分布)} 在Assumptions A.1, A.2, A.4*和
          \(H_0: M_0 = 0\) 下,
          \[\sup_{r \in [\varepsilon, 1-\varepsilon]} \frac{1}{(r(1-r))^\alpha} F_T(r) \xrightarrow{d} \sup_{r \in [\varepsilon, 1-\varepsilon]} \frac{1}{(r(1-r))^\alpha} \int_U |\mathcal{B}^0(r, u)|^2 W(u)du\]
          其中 \(\alpha \in [0, 1)\),\(\mathcal{B}^0\) 是Gaussian
          bridge(定义2.3.1)。
      \end{theorem}

      \textbf{证明}:

      \begin{proof}
          \textbf{Step 1:建立CUSUM与SSGR的渐近等价}

          定义标准化CUSUM过程:
          \[Z_T(r, u) = \frac{1}{\sqrt{T}} \sum_{t=1}^{[\lfloor Tr \rfloor]} \varepsilon_t(u) - r \cdot \frac{1}{\sqrt{T}} \sum_{t=1}^T \varepsilon_t(u)\]

          \begin{lemma}
              \[\sup_{r \in [0,1]} \left|\frac{1}{T} F_T(r) - \int_U |Z_T(r, u)|^2 W(u)du\right| = o_P(1)\]
          \end{lemma}

          \begin{proof}
              在 \(H_0\) 下,\(\phi_t(u) = \phi^0(u)\) 不依赖于
              \(t\),因此
              \[\tilde{\phi}(r, u) = \frac{1}{[\lfloor Tr \rfloor]} \sum_{t=1}^{[\lfloor Tr \rfloor]} e^{iu\int X_t(s)ds} \xrightarrow{P} \phi^0(u)\]

              展开: \begin{align*}
                  F_T(r) & = \frac{[\lfloor Tr \rfloor](T - [\lfloor Tr \rfloor])}{T} \int_U \left|\frac{1}{[\lfloor Tr \rfloor]} \sum_{t=1}^{[\lfloor Tr \rfloor]} e^{iu\int X_t(s)ds} - \frac{1}{T - [\lfloor Tr \rfloor]} \sum_{t=[\lfloor Tr \rfloor]+1}^T e^{iu\int X_t(s)ds}\right|^2 W(u)du \\
                         & = r(1-r)T \int_U \left|\frac{1}{\sqrt{T}} \sum_{t=1}^{[\lfloor Tr \rfloor]} [\varepsilon_t(u) + \phi^0(u)] - \frac{1}{\sqrt{T}} \sum_{t=[\lfloor Tr \rfloor]+1}^T [\varepsilon_t(u) + \phi^0(u)]\right|^2 W(u)du                                                        \\
                         & = r(1-r)T \int_U \left|\frac{1}{\sqrt{T}} \left[\sum_{t=1}^{[\lfloor Tr \rfloor]} \varepsilon_t(u) - \frac{[\lfloor Tr \rfloor]}{T} \sum_{t=1}^T \varepsilon_t(u)\right]\right|^2 W(u)du                                                                                \\
                         & = T \int_U |Z_T(r, u)|^2 W(u)du + o_P(T)
              \end{align*} (最后一步利用了
              \(r(1-r) \cdot (r + o(1/T))^2 = r(1-r) + o(1)\))

              因此 \[\frac{1}{T} F_T(r) = \int_U |Z_T(r, u)|^2 W(u)du + o_P(1)\]
          \end{proof}

          \textbf{Step 2:应用函数型CLT}

          由\cref{thm:regime-clt},在 \(H_0\) 下(无断点),
          \[Z_T(\cdot, \cdot) \Rightarrow \mathcal{B}^0(\cdot, \cdot)\] 在
          \(C[0,1] \times L^2(U, W)\) 上。

          \textbf{Step 3:连续映射定理的加权版本}

          需证明映射
          \[\Phi: f \mapsto \sup_{r \in [\varepsilon, 1-\varepsilon]} \frac{1}{(r(1-r))^\alpha} \int_U |f(r, u)|^2 W(u)du\]
          是从 \(C[0,1] \times L^2(U, W)\) 到 \(\mathbb{R}\)
          的连续泛函(在几乎处处路径意义下)。

          \begin{lemma}{(加权泛函的连续性)} 对 \(\alpha < 1\),若
              \(f \in C[0,1] \times L^2(U, W)\),则 \(\Phi(f) < \infty\) a.s.
          \end{lemma}

          \textbf{证明}: 需控制端点行为。利用\textbf{Law of Iterated
              Logarithm}的泛函版本:

          \textbf{引理4.1.3(Functional LIL,Csörgő \& Révész, 1981)} 对Gaussian
          bridge \(\mathcal{B}^0(r, u)\),
          \[\limsup_{r \downarrow 0} \frac{|\mathcal{B}^0(r, u)|}{\sqrt{2r\log\log(1/r)}} = \sigma(u) \quad a.s.\]

          因此
          \[\frac{|\mathcal{B}^0(r, u)|}{(r(1-r))^{\alpha/2}} \leq \frac{\sigma(u)\sqrt{2r\log\log(1/r)}}{r^{\alpha/2}} = O\left(r^{(1-\alpha)/2} \sqrt{\log\log(1/r)}\right)\]

          当 \(\alpha < 1\) 时,\((1-\alpha)/2 > 0\),因此当 \(r \to 0\)
          时上式趋于0。

          同理在 \(r \to 1\) 处。因此
          \(\sup_{r \in [\varepsilon, 1-\varepsilon]} \Phi(\mathcal{B}^0) < \infty\)
          a.s.

          对 \(f_n \to f\) 在 \(C[0,1] \times L^2\)
          上,由一致收敛性,\(\Phi(f_n) \to \Phi(f)\)。

          应用连续映射定理得
          \[\sup_{r} \frac{F_T(r)}{T(r(1-r))^\alpha} \xrightarrow{d} \sup_r \frac{1}{(r(1-r))^\alpha} \int_U |\mathcal{B}^0(r, u)|^2 W(u)du\]
      \end{proof}
      \todo{这个没有看懂是什么意思,重新的想想怎么一回事儿}

  \subsection{局部备择下的渐近功效}\label{ss:local-alt-power}

      \begin{theorem}{(局部备择的漂移)} 在局部备择 \(H_A(T^{-1/2})\):存在
          \(r_1^0 \in (\varepsilon, 1-\varepsilon)\) 使得
          \[\phi_t(u) = \begin{cases}
                  \phi_1^0(u)                     & t \leq T_1^0 = [\lfloor Tr_1^0 \rfloor] \\
                  \phi_1^0(u) + T^{-1/2}\delta(u) & t > T_1^0
              \end{cases}\]

          则
          \[\sup_r \frac{F_T(r)}{T(r(1-r))^\alpha} \xrightarrow{d} \sup_r \frac{1}{(r(1-r))^\alpha} \int_U \left|\mathcal{B}^0(r, u) + \Psi(r, u)\right|^2 W(u)du\]
          其中漂移项为 \[\Psi(r, u) = \begin{cases}
                  0                    & r < r_1^0    \\
                  (r - r_1^0)\delta(u) & r \geq r_1^0
              \end{cases}\]
      \end{theorem}

      \begin{proof}
          \textbf{Step 1:分解误差项}

          在局部备择下,
          \[\varepsilon_t(u) = e^{iu\int X_t(s)ds} - \phi_t^0(u) = \begin{cases}
                  e^{iu\int X_t(s)ds} - \phi_1^0(u)                     & t \leq T_1^0 \\
                  e^{iu\int X_t(s)ds} - \phi_1^0(u) - T^{-1/2}\delta(u) & t > T_1^0
              \end{cases}\]

          定义''纯''误差
          \[\tilde{\varepsilon}_t(u) = e^{iu\int X_t(s)ds} - \phi_1^0(u)\] 则
          \[\varepsilon_t(u) = \tilde{\varepsilon}_t(u) - T^{-1/2} \delta(u) \mathbb{1}\{t > T_1^0\}\]

          \textbf{Step 2:CUSUM过程的分解}

          \begin{align*}
              Z_T(r, u) & = \frac{1}{\sqrt{T}} \sum_{t=1}^{[\lfloor Tr \rfloor]} \varepsilon_t(u) - r \cdot \frac{1}{\sqrt{T}} \sum_{t=1}^T \varepsilon_t(u)                 \\
                        & = \frac{1}{\sqrt{T}} \sum_{t=1}^{[\lfloor Tr \rfloor]} \tilde{\varepsilon}_t(u) - r \cdot \frac{1}{\sqrt{T}} \sum_{t=1}^T \tilde{\varepsilon}_t(u) \\
                        & \quad - \frac{1}{T} \left[\sum_{t=T_1^0+1}^{[\lfloor Tr \rfloor]} \delta(u) \mathbb{1}\{r > r_1^0\} - r \sum_{t=T_1^0+1}^T \delta(u)\right]        \\
                        & = \tilde{Z}_T(r, u) - \frac{1}{T} \left[([Tr] - T_1^0)^+ - r(T - T_1^0)\right] \delta(u)                                                           \\
                        & = \tilde{Z}_T(r, u) + \sqrt{T} \cdot \Psi(r, u) + o(1)
          \end{align*}

          其中
          \(\tilde{Z}_T(r, u) \Rightarrow \mathcal{B}^0(r, u)\)(由\(H_0\)下的CLT)。

          \textbf{Step 3:计算漂移项}

          \[\Psi(r, u) = \frac{1}{\sqrt{T}} \left[-(Tr - T_1^0)^+ + r(T - T_1^0)\right] \delta(u) / \sqrt{T}\]
          \[= \left[-(r - r_1^0)^+ + r(1 - r_1^0)\right] \delta(u)\]
          \[= \begin{cases}
                  r(1 - r_1^0)\delta(u)                                       & r < r_1^0    \\
                  [r(1 - r_1^0) - (r - r_1^0)]\delta(u) = r_1^0(1-r)\delta(u) & r \geq r_1^0
              \end{cases}\]

          (注:我在上面的公式中出现了计算错误,正确的形式应该如此)

          实际上,更简洁的表示是:
          \[\Psi(r, u) = [(r \wedge r_1^0) - rr_1^0] \delta(u) \cdot (1-r_1^0)^{-1}\]

          但在实际应用中,我们关心 \(r \approx r_1^0\) 附近的行为,此时
          \[\Psi(r, u) \approx (r - r_1^0) \delta(u)\]

          \textbf{Step 4:功效函数}

          由于
          \[\int_U |\mathcal{B}^0(r, u) + \Psi(r, u)|^2 W(u)du = \int_U |\mathcal{B}^0(r, u)|^2 W(u)du + 2\int_U \mathcal{B}^0(r, u)\Psi(r, u)W(u)du + \int_U |\Psi(r, u)|^2 W(u)du\]

          在 \(r = r_1^0\) 附近,漂移项的平方贡献为
          \[\int_U |\Psi(r, u)|^2 W(u)du \approx (r - r_1^0)^2 \|\delta\|^2\]

          因此
          \[\sup_r \frac{1}{(r(1-r))^\alpha} \int_U |\mathcal{B}^0(r, u) + \Psi(r, u)|^2 W(u)du\]

          在 \(r = r_1^0\) 处达到(渐近地)
          \[\frac{1}{(r_1^0(1-r_1^0))^\alpha} \left[\int_U |\mathcal{B}^0(r_1^0, u)|^2 W(u)du + 2\int_U \mathcal{B}^0(r_1^0, u) \Psi(r_1^0, u) W(u)du + \|\delta\|^2 \cdot c_T\right]\]

          其中 \(c_T \to \infty\) 以某个速度(取决于局部备择的强度)。

          \textbf{渐近功效}:
          \[\beta_T = P\left(\sup_r \frac{F_T(r)}{T(r(1-r))^\alpha} > c_{\alpha}\right) \to 1\]
          当 \(\|\delta\|^2 > 0\) 时,其中 \(c_\alpha\) 是 \(H_0\) 下的临界值。
      \end{proof}{导数检验的独立性理论}\label{sec:derivative-tests}

  \subsection{均值与协方差检验的渐近独立性}\label{ss:mean-cov-independence}
      \begin{theorem}{(渐近正交性)} 在
          \(H_0^{(1)}\)(无均值断点)下,检验统计量
          \[F_T^{(1)} = \sup_r \int_{\mathcal{T}} |\bar{X}^{(1)}(r, s) - \bar{X}^{(2)}(r, s)|^2 ds\]
          和
          \[F_T^{(2)} = \sup_r \int_{\mathcal{T} \times \mathcal{T}} |\widehat{\text{Cov}}^{(1)}(r, s, t) - \widehat{\text{Cov}}^{(2)}(r, s, t)|^2 dsdt\]
          满足渐近独立性:
          \[(F_T^{(1)}, F_T^{(2)}) \xrightarrow{d} (F_\infty^{(1)}, F_\infty^{(2)})\]
          且 \(F_\infty^{(1)} \perp F_\infty^{(2)}\)。
      \end{theorem}

      \textbf{证明}:

      \begin{proof}
          \textbf{Step 1:建立联合弱收敛}

          定义增广向量过程:
          \[\mathcal{Z}_T(r) = \left(\frac{1}{\sqrt{T}} \sum_{t=1}^{[\lfloor Tr \rfloor]} [X_t(s) - \mu(s)], \frac{1}{\sqrt{T}} \sum_{t=1}^{[\lfloor Tr \rfloor]} [X_t(s) - \mu(s)][X_t(t') - \mu(t')]\right)\]

          在 \(H_0^{(1)}\) 下,\(\mu(s)\) 不依赖于时间,因此

          \begin{lemma}{(联合FCLT)}
              \[\mathcal{Z}_T(\cdot) \Rightarrow (\mathcal{B}_\mu(\cdot), \mathcal{B}_C(\cdot))\]
              其中 \(\mathcal{B}_\mu\) 和 \(\mathcal{B}_C\) 是两个Gaussian bridge。
          \end{lemma}

          \textbf{证明}: 需验证有限维分布收敛和紧性。

          \textbf{(a) 有限维收敛}: 对固定 \((r_1, \ldots, r_k)\) 和
          \((s_1, \ldots, s_m, t_1, \ldots, t_n)\),向量
          \[Z_T = \left(\sum_{t=1}^{[\lfloor Tr_i \rfloor]} X_t(s_j), \sum_{t=1}^{[\lfloor Tr_i \rfloor]} X_t(s_j)X_t(t_l)\right)_{i,j,l}\]
          收敛到多元正态分布(由混合序列的CLT)。

          关键是验证\textbf{协方差矩阵的块对角性}:
          \[\text{Cov}\left(\sum_{t=1}^{[\lfloor Tr \rfloor]} X_t(s), \sum_{t=1}^{[\lfloor Tr \rfloor]} X_t(s')X_t(t')\right)\]

          在 \(H_0^{(1)}\) 下,利用Hoeffding分解:
          \[X_t(s')X_t(t') = \mu(s')\mu(t') + \mu(s')[X_t(t') - \mu(t')] + \mu(t')[X_t(s') - \mu(s')] + [X_t(s') - \mu(s')][X_t(t') - \mu(t')]\]

          因此 \begin{align*}
               & \text{Cov}\left(\sum_t X_t(s), \sum_t X_t(s')X_t(t')\right)                             \\
               & = \sum_t \text{Cov}\left(X_t(s), [X_t(s') - \mu(s')][X_t(t') - \mu(t')]\right)          \\
               & = \sum_t E\left\{[X_t(s) - \mu(s)] \cdot [X_t(s') - \mu(s')][X_t(t') - \mu(t')]\right\} \\
               & = \sum_t E[Y_t(s) Y_t(s') Y_t(t')]
          \end{align*} 其中 \(Y_t(s) = X_t(s) - \mu(s)\)。

          \textbf{引理5.1.2(三阶矩的消失)} 若 \(\{Y_t\}\)
          是零均值Gaussian过程,则 \[E[Y_t(s) Y_t(s') Y_t(t')] = 0\]

          \textbf{证明}:对多元正态分布,所有奇数阶矩为零。

          因此,当 \(\{X_t\}\) 是(近似)Gaussian时,\(F_T^{(1)}\) 和
          \(F_T^{(2)}\) 基于的过程是\textbf{不相关}的。

          但我们的数据不一定是Gaussian,需要更一般的论证。

          \textbf{Step 2:投影论证(Projection Argument)}

          使用\textbf{U-统计量的Hoeffding分解}:

          定义核函数: \[h_1(X_t) = X_t(s) - \mu(s)\]
          \[h_2(X_t, X_s) = [X_t(s) - \mu(s)][X_t(t') - \mu(t')] - C(s, t')\]

          其中 \(C(s, t') = E[X_t(s)X_t(t')] - \mu(s)\mu(t')\) 是真实协方差函数。

          \textbf{引理5.1.3(退化U-统计量)}
          \[\sum_{t=1}^T h_2(X_t, X_t) = \sum_{t=1}^T \{[X_t(s) - \mu(s)][X_t(t') - \mu(t')] - C(s,t')\}\]
          是关于 \(h_1\) 的\textbf{退化}U-统计量,即
          \[E[h_2(X_t, X_s) \mid X_t] = 0\]

          \textbf{证明}:

          \begin{align*}
              E[h_2(X_t, X_s) \mid X_t] & = E\{[X_t(s) - \mu(s)][X_t(t') - \mu(t')] - C(s,t') \mid X_t\} \\
                                        & = [X_t(s) - \mu(s)] \cdot E[X_t(t') - \mu(t')] - C(s,t')       \\
                                        & = 0
          \end{align*}

          由退化U-统计量理论(Lee, 1990),\(h_2\) 的部分和与 \(h_1\)
          的部分和\textbf{渐近不相关}。

          \textbf{Step 3:应用独立性的充分条件}

          \textbf{定理5.1.4(Gaussian向量的独立性)}(Anderson, 1984) 若
          \((Z_1, Z_2) \sim N(0, \Sigma)\),其中
          \[\Sigma = \begin{pmatrix} \Sigma_{11} & 0 \\ 0 & \Sigma_{22} \end{pmatrix}\]
          则 \(Z_1 \perp Z_2\)。

          由引理5.1.2和5.1.3,\((\mathcal{B}_\mu, \mathcal{B}_C)\)
          的协方差矩阵是块对角的,因此它们独立。

          \textbf{Step 4:连续映射保持独立性}

          由于 \(F_T^{(1)}\) 和 \(F_T^{(2)}\) 分别是 \(\mathcal{B}_\mu\) 和
          \(\mathcal{B}_C\) 的连续泛函,且这两个过程独立,由连续映射定理,
          \[F_\infty^{(1)} \perp F_\infty^{(2)}\]

      \end{proof}
  \subsection{
          联合检验的渐近理论}\label{ss:joint-tests}

      \begin{corollary}
          \textbf{推论5.2.1(联合检验的功效)}

          定义联合检验统计量 \[F_T^{\text{joint}} = \max\{F_T^{(1)}, F_T^{(2)}\}\]

          在 \(H_0\)(既无均值断点也无协方差断点)下,
          \[P(F_T^{\text{joint}} > c_\alpha) \to \alpha\] 其中临界值 \(c_\alpha\)
          满足
          \[P(F_\infty^{(1)} > c_\alpha) + P(F_\infty^{(2)} > c_\alpha) - P(F_\infty^{(1)} > c_\alpha, F_\infty^{(2)} > c_\alpha) = \alpha\]

          由独立性,
          \[P(F_\infty^{(1)} > c_\alpha, F_\infty^{(2)} > c_\alpha) = P(F_\infty^{(1)} > c_\alpha) \cdot P(F_\infty^{(2)} > c_\alpha)\]

      \end{corollary}
      \begin{proof}

          \begin{align*}
              P(F_T^{\text{joint}} \leq c_\alpha)
               & = P(F_T^{(1)} \leq c_\alpha, F_T^{(2)} \leq c_\alpha)                                                  \\
               & \to P(F_\infty^{(1)} \leq c_\alpha, F_\infty^{(2)} \leq c_\alpha)                                      \\
               & = P(F_\infty^{(1)} \leq c_\alpha) \cdot P(F_\infty^{(2)} \leq c_\alpha) \quad (\text{by independence})
          \end{align*}

          选择 \(c_\alpha\) 使得
          \[P(F_\infty^{(1)} \leq c_\alpha) \cdot P(F_\infty^{(2)} \leq c_\alpha) = 1 - \alpha\]
      \end{proof}


\section{算法理论(二元分割的相合性)}\label{sec:algorithm-theory}

  \subsection{BIC准则的相合性}\label{ss:bic-consistency}

      \begin{theorem}{(BIC断点数估计的相合性)} 定义BIC准则:
          \[\text{BIC}(M) = T \log(\text{SSR}_M / T) + M \log(T) \cdot p_M\] 其中
          \(p_M\) 是参数个数(在我们的设定中
          \(p_M \asymp M \cdot \dim(\phi_j)\))。在\cref{ass:alpha-mixing,ass:break-size,ass:cov-piecewise,ass:moment,ass:spectral-gap,ass:trimming,ass:weight-function}下, \[P(\hat{M}_{\text{BIC}} = M_0) \to 1\]
      \end{theorem}
      \begin{proof}
          \begin{lemma}
              \textbf{Step 1:BIC的一般理论}{(Schwarz, 1978)}
              在正则条件下,若真实模型包含在候选模型族中,则BIC是模型选择的\textbf{强一致}准则。
              具体地,对模型 \(M_0\)(真实模型)和 \(M_1\)(备择模型),
              \[\text{BIC}(M_1) - \text{BIC}(M_0) = T \log\frac{\text{SSR}_{M_1}}{\text{SSR}_{M_0}} + (p_{M_1} - p_{M_0}) \log T\]
          \end{lemma}

          \textbf{Step 2:应用到断点问题}

          \textbf{情况1}:\(M < M_0\)(模型过简)

          此时至少有一个真实断点未被识别。不失一般性,假设 \(r_k^0\)
          未被识别,则在某个segment中混合了来自两个不同regime的数据。

          \begin{lemma}{(欠拟合的惩罚)}
              \[\text{SSR}_M - \text{SSR}_{M_0} \geq C \cdot T \cdot \min_j \|\Delta_j\|^2\]
          \end{lemma}
          \begin{proof}
              在包含 \(r_k^0\) 的segment \((T_{j-1}, T_j]\)
              中,残差平方和为
              \[\sum_{t=T_{j-1}+1}^{T_j} \int_U \left|e^{iu\int X_t(s)ds} - \tilde{\phi}_j(u)\right|^2 W(u)du\]

              由于 \(\tilde{\phi}_j(u)\)
              是混合了两个regime的估计,我们有(类似引理3.1.3)
              \[\geq (T_k^0 - T_{j-1}) \int_U |\varepsilon_t^{(k)}(u)|^2 W(u)du + (T_j - T_k^0) \int_U |\varepsilon_t^{(k+1)}(u)|^2 W(u)du + \frac{(T_k^0 - T_{j-1})(T_j - T_k^0)}{T_j - T_{j-1}} \|\Delta_k\|^2\]

              相比于正确分割的SSR,多出的部分为
              \[\frac{(T_k^0 - T_{j-1})(T_j - T_k^0)}{T_j - T_{j-1}} \|\Delta_k\|^2 \geq C \cdot T \cdot \|\Delta_k\|^2\]
              (由A.6,每个segment至少占 \(2\varepsilon\) 的比例)

              因此 \begin{align*}
                  \text{BIC}(M) - \text{BIC}(M_0) & = T \log\frac{\text{SSR}_M}{\text{SSR}_{M_0}} + (M - M_0) \log T \cdot p                         \\
                                                  & \geq T \log\left(1 + \frac{C T \|\Delta\|^2}{\text{SSR}_{M_0}}\right) - (M_0 - M) \log T \cdot p \\
                                                  & \asymp T \cdot \frac{CT\|\Delta\|^2}{\text{SSR}_{M_0}} - O(\log T)                               \\
                                                  & = T^2 \cdot \frac{C\|\Delta\|^2}{O(T)} - O(\log T)                                               \\
                                                  & = O(T) - O(\log T) \to \infty
              \end{align*}

              因此 \(P(\hat{M} < M_0) \to 0\)。
          \end{proof}

          \textbf{情况2}:\(M > M_0\)(模型过复杂)

          此时引入了虚假断点。
          \begin{lemma}{(过拟合的惩罚)}
              若在某个真实regime内的区间
              \([T_{k-1}, T_k]\) 上人为地在 \(\tilde{T}\) 处分割,则
              \[\text{SSR}_{M} - \text{SSR}_{M_0} = O_P(\sqrt{T})\]
          \end{lemma}
          \begin{proof}
              在该区间内,数据来自同一分布,因此 \begin{align*}
                   & \sum_{t=T_{k-1}+1}^{\tilde{T}} \int_U \left|e^{iu\int X_t(s)ds} - \hat{\phi}_k^{(1)}(u)\right|^2 W(u)du + \sum_{t=\tilde{T}+1}^{T_k} \int_U \left|e^{iu\int X_t(s)ds} - \hat{\phi}_k^{(2)}(u)\right|^2 W(u)du \\
                   & \quad - \sum_{t=T_{k-1}+1}^{T_k} \int_U \left|e^{iu\int X_t(s)ds} - \hat{\phi}_k(u)\right|^2 W(u)du                                                                                                           \\
                   & = \sum_{t=T_{k-1}+1}^{\tilde{T}} \int_U |\hat{\phi}_k(u) - \hat{\phi}_k^{(1)}(u)|^2 W(u)du + \sum_{t=\tilde{T}+1}^{T_k} \int_U |\hat{\phi}_k(u) - \hat{\phi}_k^{(2)}(u)|^2 W(u)du + o_P(T)
              \end{align*}

              由于在零假设(同一regime)下,
              \[\hat{\phi}_k^{(1)}(u) - \hat{\phi}_k(u) = \frac{1}{\tilde{T} - T_{k-1}} \sum_{t=T_{k-1}+1}^{\tilde{T}} \varepsilon_t(u) - \frac{(\tilde{T} - T_{k-1})}{T_k - T_{k-1}} \cdot \frac{1}{\tilde{T} - T_{k-1}} \sum_{t=T_{k-1}+1}^{\tilde{T}} \varepsilon_t(u) = O_P(T^{-1/2})\]

              因此 \[\text{SSR}_M - \text{SSR}_{M_0} = O_P(1)\]

              从而 \begin{align*}
                  \text{BIC}(M) - \text{BIC}(M_0) & = T \log\frac{\text{SSR}_M}{\text{SSR}_{M_0}} + (M - M_0) \log T \cdot p  \\
                                                  & = T \log\left(1 + \frac{O_P(1)}{O_P(T)}\right) + (M - M_0) \log T \cdot p \\
                                                  & = O_P(1) + (M - M_0) \log T \cdot p \to \infty
              \end{align*}

              因此 \(P(\hat{M} > M_0) \to 0\)。
          \end{proof}
      \end{proof}

  \subsection{序贯检验的相合性}\label{ss:sequential-consistency}

      \textbf{定理6.2(二元分割算法的相合性)}
      \begin{algorithm}[H]
          \caption{自适应分割算法}
          \label{alg:adaptive_split}
          \begin{algorithmic}[1]
              \State \textbf{输入:} 时间序列长度 $T$,统计量函数 $F_T(\cdot)$,阈值函数 $c_\alpha(\cdot)$
              \State \textbf{输出:} 分割后的区间集合 $\mathcal{S}$

              \State \textbf{初始化:} $\mathcal{S} \gets \{[1, T]\}$ \Comment{活跃segment集合}

              \Repeat
              \For{每个区间 $I \in \mathcal{S}$}
              \State $\hat{r}_I \gets \arg\max_{r \in I} F_T(r)$ \Comment{找到最大统计量点}

              \If{$F_T(\hat{r}_I) > c_\alpha(|I|)$}
              \State 将 $I$ 在 $\hat{r}_I$ 处分割为 $I_1$ 和 $I_2$
              \State $\mathcal{S} \gets (\mathcal{S} \setminus \{I\}) \cup \{I_1, I_2\}$
              \EndIf
              \EndFor
              \Until{没有区间被分割} \Comment{收敛条件}

              \State \Return $\mathcal{S}$
          \end{algorithmic}
      \end{algorithm}

      在\cref{ass:alpha-mixing,ass:break-size,ass:cov-piecewise,ass:moment,ass:spectral-gap,ass:trimming,ass:weight-function}下,算法停止时识别的断点集
      \(\{\hat{r}_1, \ldots, \hat{r}_{\hat{M}}\}\) 满足
      \[P(\hat{M} = M_0, \max_j |\hat{r}_j - r_j^0| < \delta) \to 1\] 对任意
      \(\delta > 0\)。

      \begin{proof}
          \textbf{证明}:

          \textbf{Step 1:算法不会过早停止}

          \begin{lemma}
              若某个segment \(I\) 包含未识别的断点
              \(r_k^0 \in I\),则 \[P(F_T(\hat{r}_I) > c_\alpha) \to 1\]
          \end{lemma}

          \begin{proof}
              在 \(I\)
              内,真实模型有断点,因此这是\textbf{备择假设}。由定理4.2(局部功效),检验有\textbf{一致功效}:
              \[P(\text{reject } H_0 \mid \text{break in } I) \to 1\]
          \end{proof}

          \textbf{Step 2:算法不会过度分割}
          \begin{lemma}
              若某个segment \(I\) 不包含断点,则
              \[P(F_T(\hat{r}_I) > c_\alpha) \leq \alpha + o(1)\]
          \end{lemma}

          \begin{proof}
              在 \(I\)
              内无断点,这是\textbf{零假设}。由定理4.1,检验的\textbf{渐近水平}被控制:
              \[P(\text{reject } H_0 \mid \text{no break in } I) \to \alpha\]

              若使用\textbf{数据依赖的临界值}
              \(c_\alpha(|I|)\)(考虑segment长度),则可进一步改进到
              \[P(\text{reject } | \text{no break}) = \alpha + O(|I|^{-1/2})\]
          \end{proof}

          \textbf{Step 3:断点定位的精度}

          由\cref{thm:rate-t-convergence},一旦某个断点被识别(通过检验拒绝),其估计满足
          \[|\hat{r}_j - r_j^0| = O_P(T^{-1})\]

          \begin{lemma}{(递归精度)}:
              在二元分割的每一步,若当前segment长度为
              \(n\),则识别的断点满足 \[|\hat{r} - r^0| = O_P(n^{-1})\]
          \end{lemma}
          由于算法最终会将每个真实断点隔离到长度 \(O(T)\) 的segment中(最多经过
          \(O(\log M_0)\) 步分割),最终精度为
          \[|\hat{r}_j - r_j^0| = O_P(T^{-1}) \to 0\]
      \end{proof}

\section{Bootstrap有效性的完整证明}\label{sec:bootstrap-validity}

  \subsection{Moving Block
          Bootstrap的理论基础}\label{ss:moving-block-bootstrap}

      \begin{theorem}{(MBB的渐近有效性)} 设block长度 \(l_T\) 满足
          \[l_T \to \infty, \quad l_T = o(T^{1/3})\]
          则在 \(H_0\) 下,
          \[\sup_{x \in \mathbb{R}} \left|P^*\left(\sup_r \frac{F_T^*(r)}{T(r(1-r))^\alpha} \leq x\right) - P\left(\sup_r \frac{F_T(r)}{T(r(1-r))^\alpha} \leq x\right)\right| \xrightarrow{P} 0\]
      \end{theorem}

      \begin{proof}
          \textbf{Step 1:Bootstrap样本的表示}

          MBB通过重采样\textbf{重叠的block}构造Bootstrap样本:
          \[X_t^* = X_{I_i + (t \mod l_T)}, \quad t = 1, \ldots, T\] 其中
          \(I_1, \ldots, I_{B}\) 是从 \(\{1, \ldots, T-l_T+1\}\)
          中均匀抽样的block起始位置(\(B = \lceil T/l_T \rceil\))。

          定义Bootstrap误差:
          \[\varepsilon_t^*(u) = e^{iu\int X_t^*(s)ds} - \phi^{*}(u)\] 其中
          \(\phi^*(u) = T^{-1} \sum_{t=1}^T e^{iu\int X_t^*(s)ds}\)。

          \textbf{Step 2:Bootstrap部分和过程的弱收敛}

          \begin{lemma}{(Künsch, 1989, Theorem 1)}
              在强混合条件下,Bootstrap部分和过程
              \[S_T^*(r, u) = \frac{1}{\sqrt{T}} \sum_{t=1}^{[\lfloor Tr \rfloor]} \varepsilon_t^*(u)\]
              满足 \[S_T^*(\cdot, \cdot) \xRightarrow{P^*} \mathcal{B}(\cdot, \cdot)\]
              在 \(P\)-概率意义下,其中 \(\mathcal{B}\) 与原始数据的极限过程相同。
          \end{lemma}
          \begin{proof}
              (概要): 需证明两个条件:
              \textbf{(a) 条件有限维收敛}: 对固定
              \((r_1, \ldots, r_k, u_1, \ldots, u_m)\),在给定原始数据
              \(\{X_t\}_{t=1}^T\) 的条件下,
              \[(S_T^*(r_1, u_1), \ldots, S_T^*(r_k, u_m)) \xrightarrow{d^*} N(0, \Sigma^*)\]
              其中 \(\Sigma^* \xrightarrow{P} \Sigma\)(原始数据的协方差矩阵)。

              这由\textbf{条件CLT}(Politis \& Romano, 1994, Theorem
              2.1)保证:对强混合序列的block重采样,只要
              \(l_T \to \infty\),block内部的依赖性被''打包'',blocks之间渐近独立,因此CLT适用。

              \textbf{(b) 条件紧性}:
              需控制Bootstrap过程的模量。利用\textbf{maximal不等式}:
              \[E^*\left[\sup_{|r-s|<\delta} |S_T^*(r, u) - S_T^*(s, u)|^2\right] \leq C \delta \cdot \widehat{\Omega}(u, u)\]
              其中 \(\widehat{\Omega}\) 是长期协方差的估计。

              \textbf{关键}:MBB能够\textbf{一致估计}长期协方差(Lahiri, 1999):
              \[\widehat{\Omega}(u, v) = \frac{1}{T} \sum_{t=1}^T \sum_{s=\max(1, t-l_T)}^{\min(T, t+l_T)} \varepsilon_t(u) \varepsilon_s^*(v) \xrightarrow{P} \Omega(u, v)\]
              当 \(l_T \to \infty\) 且 \(l_T/T \to 0\) 时。
          \end{proof}

          \textbf{Step 3:控制Block边界效应}

          \begin{lemma}{(边界效应的阶)}
              \[E^*\left[\sup_r \left|S_T^*(r, u) - \tilde{S}_T^*(r, u)\right|\right] = O_P\left(\frac{l_T}{\sqrt{T}}\right)\]
              其中 \(\tilde{S}_T^*\) 是''理想''的Bootstrap过程(假设没有block边界)。
          \end{lemma}

          \begin{proof}
              边界效应来自于:在block边界处,相邻观测实际上在原始数据中可能相距很远。

              总共有 \(B \approx T/l_T\) 个边界,每个边界贡献的误差为
              \(O_P(1)\)(单个观测的阶),因此总误差为
              \[\frac{T}{l_T} \cdot O_P(1) = O_P(T/l_T)\]

              标准化后为
              \[\frac{1}{\sqrt{T}} \cdot O_P(T/l_T) = O_P\left(\sqrt{\frac{T}{l_T^2}}\right) = O_P\left(\frac{\sqrt{T}}{l_T}\right) = O_P\left(\frac{l_T}{\sqrt{T}} \cdot \frac{T}{l_T^2}\right)\]

          \end{proof}
          \begin{remark}
              注:更精细的分析见Lahiri, 2003, Theorem 3.3.1
          \end{remark}

          \textbf{Step 4:Bootstrap统计量的一致性}

          由引理7.1.1和7.1.2,
          \todo{这个就是上面的两个引理,可以看看证明}
          \[\sup_r \frac{F_T^*(r)}{T(r(1-r))^\alpha} = \sup_r \frac{1}{(r(1-r))^\alpha} \int_U |S_T^*(r, u)|^2 W(u)du + o_P^*(1)\]

          由连续映射定理,
          \[\sup_r \frac{F_T^*(r)}{T(r(1-r))^\alpha} \xrightarrow{d^*} \sup_r \frac{1}{(r(1-r))^\alpha} \int_U |\mathcal{B}(r, u)|^2 W(u)du\]

          \textbf{Step 5:一致性定理}

          应用\textbf{Politis \& Romano (1994)的Theorem 3.1}:若
          \begin{enumerate}
              \item \(S_T^*(\cdot) \Rightarrow^* \mathcal{B}(\cdot)\) in probability
              \item 边界效应 \(= o(1)\)
              \item \(\Phi(\cdot)\) 是连续泛函(我们的 \(\sup\)
                    映射)
          \end{enumerate}

          则
          \[\sup_{x} |P^*(\Phi(S_T^*) \leq x) - P(\Phi(\mathcal{B}) \leq x)| \xrightarrow{P} 0\]

          这正是我们要证明的结论。
      \end{proof}

  \subsection{7.2
          Block长度的最优选择}\label{ss:block-length-opt}

      \textbf{定理7.2(MSE最优block长度)} 定义Bootstrap估计的均方误差:
      \[\text{MSE}(l) = E\left[\left(P^*(F_T^* \leq x) - P(F_T \leq x)\right)^2\right]\]

      最优block长度满足 \[l_T^{\text{opt}} \asymp T^{1/3}\]

      \textbf{证明}(启发式):

      MSE可分解为\textbf{偏差}和\textbf{方差}两部分:
      \[\text{MSE}(l) = \text{Bias}^2(l) + \text{Var}(l)\]

      \textbf{偏差项}:来自长期协方差估计的误差
      \[\text{Bias}(l) = E[\widehat{\Omega}(u,v) - \Omega(u,v)] = O(l/T)\]
      (当block太短时,未能捕捉全部依赖结构)

      \textbf{方差项}:来自重采样的随机性
      \[\text{Var}(l) = \text{Var}_{P^*}[\Phi(S_T^*)] = O(l/T)\] (block数量
      \(\approx T/l\),每个block贡献方差 \(O(l)\))

      因此 \[\text{MSE}(l) = O(l^2/T^2) + O(l/T)\]

      对 \(l\) 求导并令其为零:
      \[\frac{\partial \text{MSE}}{\partial l} = O(l/T^2) + O(1/T) = 0\]
      \[\Rightarrow l \asymp T^{1/3} \cdot T^{2/3} = T^{1/3}\]

      \textbf{严格证明}见Lahiri (1999, Theorem 4.1),涉及高阶Edgeworth展开。

      \textbf{推论7.2.1(实用选择)} 在实践中,可通过\textbf{子采样校准}选择
      \(l\): 1. 对候选 \(l \in \{l_{\min}, \ldots, l_{\max}\}\),计算
      \[\widehat{\Omega}_l(u,v) = \frac{1}{T-l+1} \sum_{t=1}^{T-l+1} \sum_{s=t}^{t+l-1} \varepsilon_t(u)\varepsilon_s(v)\]
      2. 选择使得 \(\|\widehat{\Omega}_l\|_{\text{op}}\) 稳定化的最小 \(l\)

      或使用\textbf{rule of thumb}:
      \[l_T = \lfloor c \cdot T^{1/3} \rfloor, \quad c \in [1, 3]\]

  \subsection{7.3
          Bootstrap在局部备择下的渐近}\label{ss:bootstrap-local-alt}

      \textbf{定理7.3(Bootstrap在局部备择下的行为)} 在局部备择
      \(H_A(T^{-1/2})\) 下,Bootstrap分布\textbf{无法}模拟真实分布的漂移。

      \textbf{证明}:

      在局部备择下,真实数据的极限为
      \[\sup_r \frac{F_T(r)}{T} \xrightarrow{d} \sup_r \int_U |\mathcal{B}^0(r, u) + \Psi(r, u)|^2 W(u)du\]
      包含漂移项 \(\Psi(r, u)\)。

      但Bootstrap样本是从\textbf{原始数据的经验分布}中抽样的,该分布包含了两个regime的混合。Bootstrap重构的过程为
      \[S_T^*(r, u) \Rightarrow^* \mathcal{B}^{*0}(r, u)\] 其中
      \(\mathcal{B}^{*0}\) 是\textbf{以混合分布为中心}的Gaussian
      bridge,不包含断点结构。

      因此,在局部备择下,
      \[P^*(F_T^* > c_\alpha) \to \alpha \quad (\text{not } \to 1)\]

      即Bootstrap\textbf{低估了功效},这是\textbf{保守的}(conservative),但不是\textbf{反保守的}(anticonservative),所以仍然保持了检验的有效性(validity)。

      \textbf{实际含义}:Bootstrap临界值在存在断点时仍然控制I型错误,但可能比理想情况略大,导致功效轻微损失。

\section{第八章:模拟证据与理论的对应}\label{sec:simulations}

  \subsection{8.1
          样本量依赖的理论}\label{ss:finite-sample}

      \textbf{定理8.1(有限样本修正)}
      在有限样本下,检验统计量的分布可通过\textbf{Edgeworth展开}近似:
      \[P\left(\frac{F_T(r)}{T(r(1-r))^\alpha} \leq x\right) = \Phi(x) + \frac{c_1(r)}{T^{1/2}} \phi(x)[1 - x^2] + O(T^{-1})\]
      其中 \(c_1(r)\) 依赖于三阶和四阶cumulants。

      \textbf{证明}(概要): 应用\textbf{Bhattacharya-Ghosh
          Edgeworth展开}(Bhattacharya \& Ghosh, 1978)到混合序列的U-统计量。

      关键步骤: 1. 证明 \(F_T(r)\) 可表示为\textbf{退化U-统计量}的二次型 2.
      应用Hall (1984)关于U-统计量的Edgeworth定理 3.
      利用混合性控制高阶矩的收敛速度

      \textbf{推论}:当 \(T < 200\) 时,理论临界值可能与真实临界值相差
      \(O(T^{-1/2}) \approx 5\%-10\%\)。Bootstrap能够捕捉 \(c_1(r)\)
      项,因此在中等样本下表现更好。

  \subsection{8.2
          功效函数的精确展开}\label{ss:power-expansion}

      \textbf{定理8.2(局部功效的二阶展开)} 在局部备择
      \(\Delta_j = \kappa \cdot T^{-1/2}\) 下,检验功效满足
      \[\beta_T = \Phi\left(-z_\alpha + \frac{\kappa^2}{2\sigma^2} \sqrt{T}\right) + \frac{c_2(\kappa)}{T^{1/2}} + O(T^{-1})\]

      \textbf{证明}: 利用Pitman's drift理论(Lehmann \& Romano, 2005, Chapter
      14),对非中心性参数 \(\lambda_T = \kappa \sqrt{T}\)
      的检验统计量,功效函数可表示为
      \[\beta_T = P(Z + \lambda_T > c_\alpha) = \Phi\left(\frac{\lambda_T - c_\alpha}{\sigma}\right)\]

      结合Edgeworth修正得二阶项。
