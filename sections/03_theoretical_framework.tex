\section{Theoretical Framework}
\label{sec:theory}

This section presents the theoretical framework underlying the econometric analysis.

% 基本模型设定
\subsection{Model Setup}

Consider the following model:
\begin{equation}
    Y_i = \beta_0 + \beta_1 X_i + \varepsilon_i
    \label{eq:basic_model}
\end{equation}
where $Y_i$ is the outcome variable, $X_i$ is the treatment variable, and $\varepsilon_i$ is the error term.

% 假设条件
\subsection{Assumptions}

\begin{assumption}[Linearity]
    The relationship between $Y_i$ and $X_i$ is linear in parameters.
    \label{assump:linearity}
\end{assumption}

\begin{assumption}[Exogeneity]
    The error term is uncorrelated with the explanatory variables:
\end{assumption}

% 理论推导
\subsection{Theoretical Results}

\begin{theorem}[Main Theorem]
    Under Assumptions \ref{assump:linearity} and \ref{assump:exogeneity}, the estimator $\hat{\beta}$ is consistent.
    \label{thm:main}
\end{theorem}

\begin{proof}
    The proof follows from...
\end{proof}
